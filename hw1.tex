%%This is a very basic article template.
%%There is just one section and two subsections.


\documentclass{article}
\usepackage{mathtools}  % loads »amamsth«
\usepackage{amssymb}
\usepackage{dsfont}

\title{Assignment 1}
\author{Amit Wolfenfeld}
\date{17 November 2013}


\begin{document}
\maketitle

\section*{Question 1}
First let us define the payoff matrix:
\\

$$ U = \begin{pmatrix}

1,1	& 1,2 & 1,3  & \dots 1,49 & 1,50 & 1,51  & \dots   &  1,98	& 1,99 & 1,99  \\

2,1	& 2,2 & 2,3  & \dots 2,49 & 2,50 & 2,51  & \dots   &  2,98	& 2,98 & 2,98  \\

3,1	& 1,2 & 1,3  & \dots 3,49 & 3,50 & 3,51  & \dots   &  3,97	& 3,97 & 3,97  \\

\vdots  & \dots & \dots & \ddots & \vdots & \dots & \dots & \dots & \dots \\

49,1	& 49,2 & 49,3  & \dots 49,49 & 49,50 & 49,51  & \dots   &  49,51	& 49,51 &
49,51
\\

50,1	& 50,2 & 50,3  & \dots 50,49 & 50,50 & 50,50  & \dots   &  50,50	& 50,50 &
50,50
\\

51,1	& 51,2 & 51,3  & \dots 51,49 & 50,50 & 50,50  & \dots   &  51,49	& 51,49 &
51,49
\\

\vdots  & \dots & \dots & \dots & \vdots & \dots & \ddots & \dots & \dots \\

98,1	& 98,2 & 97,3  & \dots 51,49 & 50,50 & 49,51  & \dots    & 50,50	& 98,2 &
98,2
\\

99,1	& 98,2 & 97,3  & \dots 51,49 & 50,50 & 49,51  & \dots   & 2,98	& 50,50 &
99,1
\\

99,1	& 98,2 & 97,3  & \dots 51,49 & 50,50 & 49,51  & \dots   &  2,98	& 1,99 &
50,50
\\

\end{pmatrix}
$$ 

Let us start with Player 1, It is obvious that row 2 dominates row 1, hence we
can cross out row 1. It is also obvious that row 3 dominates row 2, hence we can
cross out that one as well. We can continue with this till row 50 which
dominates row 49. How ever row 51 does not dominated row 50.
We can do the same for Player 2 with the columns till we reach a sub matrix of
51 by 51.\\

%In this new game row 51 weakly dominates row 50 so we can cross it out. Same
%for collumn 51 and 50. At this point we are left with a subgame of 50 by 50. \\

%In this new game row 51 weakly dominates row 100 so we can cross it 49 by 49.
%\\

%In this new game row 51 weakly dominates row 99 so we can cross it out. Same
%for collumn 51 and 99. At this point we are left with a subgame of 48 by 48.
%\\

At this point we don't have any dominating strategies so we will look for the
best action.\\
If player 1 chooses 100 player 2 best action is 99. If player 1 chooses 99
player 2 best action is 98 and so meaning if player 1 choose $n$ player 2 best
action would be $n-1$, this holds for $ 51 \leq n \leq 100$. the same for the
player 2, for any action $n$ player 1 best action is $n-1$ for $ 51 \leq n \leq
100$. From this we find that the only NE that are relevant in the 50 by 50 game
is player one chooses 51 and player 2 chooses 51. Now let have a look at what
happens if player 1 chooses 50, player 2 best action will be any action (they
yield 50) and the other way around. This gives us that another NE is if player 1
chooses 50 and player 2 chooses 50.

To conclude, the NE's are $(50,50)$, $(51,51)$.


\section*{Question 2}

Lets consider a direct auction where all the buyers including  the seller
(auctioneer) bid.  The winning rule is that the highest bid wins, and the
payment rule is that the highest bid pays the second highest bid (same as in the second price
auction). 
\\
The difference between this auction and the second price auction is
that if the highest bid belongs to the seller, non of the buyers will get the
goods. 
\\
Lets prove that being truthful is a weakly dominant strategy:
Suppose the for buyer $i$ value is $v_i$,  and he considers biding $b_i>v_i$ or
$b_i<v_i$.
Lets denote the highest bid $\hat{b}$ of the other players including the
auctioneer/seller.
\\
\\
There are 6 possible outcomes:
\\
$b_i>v_i>\hat{b}$: The buyer will get the goods, the profit is $v_i-\hat{b}$, 
and could have got the same result bidding $b_i=v_i$.
\\
$v_i>b_i>\hat{b}$:The buyer will get the goods, the profit is $v_i-\hat{b}$,
and could have got the same result bidding $b_i=v_i$.
\\
$b_i>\hat{b}>v_i$: The buyer gets the goods but the the utility is negative,
could have done better if he had bid $b_i=v_i$.
\\
$v_i>\hat{b}>b_i$: The buyer did not get the goods and could have done better if
he had bid $v_i$ (would get a utility of $v_i-\hat{b}$ instead of 0)
\\
$\hat{b}>v_i>b_i$: The buyer did not get the goods, would  have done the same if
had bid $v_i$.
\\
$\hat{b}>b_i>v_i$: The buyer did not get the goods, would  have done the same if
had bid $v_i$.
\\
\\
To conclude the player does better if he bids his actual value $v_i$ then to bid
higher or lower. Till now we referred to the auctioneer as a player, and this is true unless his value is
the highest, when this occurs (the seller has the highest bid) he gets the goods
and no other player, so this does not affect the weakly dominating strategy of
bidding truthfully. This type of auction satisfies all the demands and disputes
the claim.

\section*{Question 3}

First let us define the utility function

$\begin{matrix} u(v_i,b_i) = \left\{ \begin{array}{ll}
         v_i-b_i & \mbox{if $b_i>max\{b_j\}$};\\
         \frac{v_i-b_i}{k} & \mbox{if k ties};\\
        0 & \mbox{else}\end{array} \right.
        
 \end{matrix}  $

We assume any $v_i$ is drawn from $f$ and has a CDF of $F$ and is $iid$.
\\
We also assume that the $b_i$ is a function of $v_i$ and that all the player use
the same policy to maximize their gain. Meaning $b_i(v) = b_j(v)$ (for the
same $v$).\\

The player wishes to maximize his gain hence he will play $b^*(v)$ that
maximizes the following:

$$Pr(win|b^*(v))(v-b^*(v)) = Pr(b^*(v)>\max\{b_j(v_j)\})(v-b^*(v))$$

Because of the $iid$ assumption we can say

$$\Pi\{Pr(b^*(v)>b_j(v_j)\}(v-b^*(v))=\{Pr(b^*(v)>b(v))\}^{n-1}(v-b^*(v))$$

We will assume that $b(v)$ is an a non-decreasing function in $v$ and we get  

$$Pr(win|b^*(v))(v-b^*(v)) = \{F(b^{-1}(b^*(v)))\}^{n-1}(v-b^*(v))$$

Lets  derive and compare to zero (assuming $F'=f$):

$$\frac{(n-1)f(b^{-1}(b^*(v)))\{F(b^{-1}(b^*(v)))\}^{n-2}}{b'(b^{-1}(b^*(v))}(v-b^*(v))
- \{F(b^{-1}(b^*(v)))\}^{n-1} = 0$$

We are looking for the equilibrium and hence $b^*(v)=b(v)$ so we can say the
following: 

$$\frac{F^{n-1}(v)}{b'(v)}(v-b(v))-F^{n-1}(v) $$
Integration by parts will give us:

$$b^*(v) = v-\frac{\int_0^v F^{n-1}(x)dx}{F^{n-1}(v)}$$

Now we need to use the CDF that is given 



$\begin{matrix} F^{n-1}(x) = \left\{ \begin{array}{ll}
         0 & \mbox{if $v<a$};\\
         (\frac{x-a}{b-a})^{n-1} & \mbox{if $a\leq x \leq b$};\\
        1 & \mbox{else}\end{array} \right.
        
 \end{matrix}  $

And after a bit of calculus we get:

For any $v$ that satisfies $a\leq v \leq b$ we get

$$ \int_0^v F^{n-1}(x)dx=\int_a^v F^{n-1}(x)dx$$



$$
\frac{(b-a)}{n}(\frac{v-a}{b-a})^{n}-\frac{(b-a)}{n}(\frac{a-a}{b-a})^{n} =
\frac{(b-a)}{n}(\frac{v-a}{b-a})^{n} = $$
$$
=\frac{1}{n}\frac{(v-a)^n}{(b-a)^{n-1}}
$$

$$
\frac{\int_0^v
F^{n-1}(x)dx}{F^{n-1}(v)}=\frac{(v-a)^{n}}{n(b-a)^{n-1}} \cdot
(\frac{b-a}{v-a})^{n-1} = \frac{v-a}{n}$$
\\

And we finally get:

$$b^*(v) = v-\frac{v-a}{n}=\frac{(n-1)v+a}{n}$$

Lets make a quick sanity check with $a=0$ like we have seen in class:

$$b^*(v) =\frac{(n-1)}{n}v
$$

yey!


\section*{Question 4}
In this question the parameters are $n=3$, $a=5$, $b=10$, $v=8$. we get 

$$b^*(v) =\frac{(n-1)v+a}{n}=\frac{(3-1)8+5}{3}=\frac{21}{3}=7$$

Now let us calculate the expected utility:

$$ \mathbb{E}_{v_2 \sim U[5,10],v_3 \sim U[5,10]} \left[ v_1-b_1 \right]$$

$$ = (v_1-b_1)\int_{5}^{10}\int_{5}^{10}
\mathds{P}(b_1>\frac{2 v_2 + 5}{3},b_1>\frac{2 v_2 + 5}{3})dv_2 dv_3$$

Now lets calculate the the expected utility for the different bids. 
\\
Lets start witht the equilibrium $b_1=7$:
$$  (8-7)\int_{5}^{10}\int_{5}^{10}
\mathds{P}(7>\frac{2 v_2 + 5}{3})\mathds{P}(7>\frac{2 v_2 + 5}{3})dv_2 dv_3$$
$$
=\int_{5}^{10}\mathds{P}(v_2<8)\int_{5}^{10}\mathds{P}(v_3<8)
=\frac{3}{5}\frac{3}{5} = \frac{9}{25} $$
\\
Now with $b_1=6$

$$  (8-6)\int_{5}^{10}\int_{5}^{10}
\mathds{P}(6>\frac{2 v_2 + 5}{3})\mathds{P}(6>\frac{2 v_2 + 5}{3})dv_2 dv_3$$
$$
=2\int_{5}^{10}\mathds{P}(v_2<6.5)\int_{5}^{10}\mathds{P}(v_3<6.5)
=2\frac{3}{10}\frac{3}{10} = 2 \frac{9}{100} = \frac{9}{50} $$

And now with $b_1 = 8$

$$  (8-8)\int_{5}^{10}\int_{5}^{10}
\mathds{P}(8>\frac{2 v_2 + 5}{3})\mathds{P}(8>\frac{2 v_2 + 5}{3})dv_2 dv_3 =0$$

It is clear that $b_1=7$ is the best bid.

\end{document}
